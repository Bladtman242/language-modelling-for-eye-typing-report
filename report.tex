\documentclass[a4paper, titlepage]{article}

\usepackage[utf8]{inputenc}
\usepackage{courier} % Required for the courier font
\usepackage[bookmarks]{hyperref}

%redefine percentage sign to be a little smaller
\let\oldpct\%
\renewcommand{\%}{\scalebox{.9}{\oldpct}}

\setcounter{secnumdepth}{0}
\begin{document}

\title{The Eye as a Security Mechanism}
\author{
	Sigurt Bladt Dinesen
	\\\texttt{sidi@itu.dk}
	\and
	Advisor:
	\\Dan Witzner Hansen
}


\maketitle
\tableofcontents
\clearpage
\section{Introduction}
The purpose of this project is to explore the use of language- and
input-models, meant to facilitate eye typing with uncalibrated eye tracking.

Typing is a ubiquitous input method for human-computer interaction, particularly for human-to-human communication facilitated by computers.

Several methods have been developed to help humans type faster and more
accurately, especially on mobile platforms like mobile phones. Some
methods are based on probabilistic language- and input-modelling, such
as T9 and the replace-as-you-type technologies (colloquially;
\emph{autocorrect}) that have mostly replaced T9 on contemporary
smartphones, perhaps due to widespread deployment of touchscreens. Other
methods use gestures, or leverage creative arrangements of input
symbols. Examples include swype for Android and IOS, and Dasher and
StarGazer intended for use with eye tracking.

The use of eye tracking as a means of text input (eye typing) can be
demanding on the user. Exact and deliberate control of the gaze is
difficult and tiring, and in addition to the energy expended when
typing, most systems require users to go through calibration routines
before use.

One area where eye typing has seen use is as a communication platform
for ALS (amyotrophic lateral sclerosis) patients. Patients suffering
suffering from ALS often retain the use of their eye muscles longer than
that of other muscles, and eye typing can therefore help increase their
quality of life. In practice, patients tend to use eye typing
for short commands and answers. For such usage patterns, uncalibrated
eye typing will be particularly beneficial.

To explore facilitation of uncalibrated tracking, the following work is
presented:
\begin{enumerate}
\item
  A language- and input-model is presented, in order
  to make educated guesses as to the intended user input, in an attempt
  to compensate for tracker (and user) imprecisions. In particular, the goal of
  the models is to ease typing by accepting mistakes, to lessen the need of
  typing, by guessing future inputs.
\item
  A typing system is implemented, as a simulation using a
  pointing device (mouse). To solve the problem known as
  \emph{midas touch}, input symbols continuously move along the perimeter of a
  circle. This movement is then correlated with the user input.
\item
  The typing system is evaluated in an informal typing-speed experiment.
\end{enumerate}

The deliverables of is project are software implementations of the
language- and input-model, of the eye typing system(s), and a report
detailing the project process and results.

\section{Background}
\subsection{Input model}
\subsection{Language model}

\section{Project Design}
\subsection{Input model}
\subsection{Language model}

\section{User study}
\section{Conclusion}

\end{document}

